% CVS info. These are modified by cvs at checkout time.
% The last version of these macros found before the maketitle will be the one on the front page,
% so only the main file is tracked.
% Edit by hand with care!
 \RCS$Revision: 1.1 $
 \RCS$Date: 2009/09/28 13:33:42 $
 \RCS$Name:  $
%%%%%%%%%%%%% ptdr definitions %%%%%%%%%%%%%%%%%%%%%
\input{ptdr-definitions}
%%%%%%%%%%%%%%%  Title page %%%%%%%%%%%%%%%%%%%%%%%%
% [Not required for PAS notes -- derived from directory name.] Please replace 2006/000 with your note number in the following line:
%\cmsNoteHeader{XXX-09-YYYY}
\title{Charge calibration of the CMS pixel detector}% Force line breaks with \\

\address[unkno]{unknown}
\address[ua]{Universiteit Antwerpen}
\address[cu]{Cornell University}
\author[cu]{Freya Blekman}
\author[ua]{Romain Rougny}
\author[unkno]{Someone Else}
%\author{The CMS Collaboration}

% please supply the date in yyyy/mm/dd format. Today has been
% redefined to do so, but it should be fixed as of the final release date.
\date{\today}

% note that you cannot use \verb in the abstract text
\abstract{
This note is intended as documentation on pixel charge calibrations at the start of LHC data taking.
}

% these need to be filled in by hand and should (MUST) match the info
% in the TeX equivalents less the TeX markup
\hypersetup{%
pdfauthor={CMS Collaboration},%
pdftitle={Test PAS Paper},%
pdfsubject={CMS},%
pdfkeywords={CMS, physics, software, computing}}
\maketitle %maketitle comes after all the front information has been supplied

%%%%%%%%%%%%%%%%%%%%%%%%%%%%%%%%  Begin text %%%%%%%%%%%%%%%%%%%%%%%%%%%%%
\section{Introduction}
Freya
\section{Technical Implementation}
\subsection{Data collection}
What is the raw data format - how is it collected at P5
Freya
\subsection{Analysis code}
\subsubsection*{Raw data}
How is the raw data processed (analysis flow)
Freya
\subsubsection*{CalibDigis}
about calibdigis 
Freya
\subsubsection*{Basic Analyzers}
intro to the loops

\subsection{Gain Analyzer}
Romain

\subsection{Analysis output}
What is the output of the analysis
Romain
\subsection{Databases}
How do we store the information in the DB? what format/granularity is used? 
Romain
\subsection{Implementation in CMSSW reconstruction}
How does the reconstruction use this information? And in which stage?
Freya
\section{Some results}
Show some 'good/bad' plots, the CRAFT09 before/after calibration plots come to mind
Romain
\section{Conclusion}
Freya

\bibliography{auto_generated}   % will be created by the tdr script.
\clearpage
\appendix

